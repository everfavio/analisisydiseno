\documentclass{beamer}
\usepackage[utf8]{inputenc}
\usepackage[spanish]{babel}

\title{Frases destacadas de "El Hombre Mes Mitológico"}
\author{ Fred Brooks - 1975}


\begin{document}

\begin{frame}
	\titlepage
\end{frame}

\begin{frame}
	\frametitle{Brooks' Law}
	\begin{itemize}
		\item \textbf{Frase:} "Añadir más gente a un proyecto de software retrasado lo hará más tarde aún."
		\item \textbf{Explicación:} Añadir más recursos humanos a un proyecto en problemas a menudo aumenta la complejidad en lugar de acelerar el progreso. La incorporación de nuevos miembros requiere tiempo para la formación y aumenta la carga de comunicación, retrasando aún más el proyecto.
	\end{itemize}
\end{frame}

\begin{frame}
	\frametitle{Construye uno para tirar}
	\begin{itemize}
		\item \textbf{Frase:} "Construye uno para tirar; de todos modos, lo harás."
		\item \textbf{Explicación:} Brooks sugiere que el primer sistema que se construye es, en esencia, un prototipo que deberás descartar. Luego podrás construir la versión real con una mejor comprensión de los problemas y requisitos, lo que resultará en un diseño más efectivo y eficiente.
	\end{itemize}
\end{frame}

\begin{frame}
	\frametitle{No hay bala de plata}
	\begin{itemize}
		\item \textbf{Frase:} "No hay bala de plata."
		\item \textbf{Explicación:} Esta frase significa que no existe una solución única y mágica que resuelva todos los problemas en el desarrollo de software. En lugar de buscar una "cura milagrosa", es mejor abordar los desafíos del desarrollo de software de manera más realista y multifacética.
	\end{itemize}
\end{frame}

\begin{frame}
	\frametitle{Complejidad en la Programación}
	\begin{itemize}
		\item \textbf{Frase:} "La complejidad de la programación se debe a que es un ejercicio de diseño explícito."
		\item \textbf{Explicación:} Brooks argumenta que el desarrollo de software es inherentemente complicado porque implica un diseño detallado y deliberado, lo cual lo diferencia de otras formas de construcción o fabricación.
	\end{itemize}
\end{frame}

\begin{frame}
	\frametitle{Programación en Solitario}
	\begin{itemize}
		\item \textbf{Frase:} "La programación en solitario es la mejor forma de abordar un proyecto pequeño de rápida evolución."
		\item \textbf{Explicación:} Aquí, Brooks destaca que, para proyectos más pequeños y ágiles, un solo programador competente puede ser más efectivo que un equipo grande, ya que elimina la sobrecarga de comunicación y coordinación.
	\end{itemize}
\end{frame}

\begin{frame}
	\frametitle{Tarifas de Error y Tamaño del Programa}
	\begin{itemize}
		\item \textbf{Frase:} "Las tarifas de error son estrechamente proporcionales al tamaño del programa."
		\item \textbf{Explicación:} Esta frase destaca la relación entre la complejidad (tamaño) del software y la cantidad de errores que se pueden esperar. Cuanto más grande y complejo es el código, más probable es que contenga errores.
	\end{itemize}
\end{frame}

\begin{frame}
	\frametitle{La Existencia del 'Alquitrán'}
	\begin{itemize}
		\item \textbf{Frase:} "El 'alquitrán' está allí porque están sucediendo cosas complicadas, y no hay forma de simplificarlas."
		\item \textbf{Explicación:} Brooks señala que ciertos aspectos de la ingeniería de software son intrínsecamente complejos y no pueden simplificarse sin perder su esencia.
	\end{itemize}
\end{frame}

\begin{frame}
	\frametitle{Importancia de la Documentación}
	\begin{itemize}
		\item \textbf{Frase:} "Documentar el diseño de un programa es una inversión de alto rendimiento."
		\item \textbf{Explicación:} Brooks enfatiza la importancia de documentar el diseño y la arquitectura del software, ya que facilita el mantenimiento y la comprensión del sistema a largo plazo.
	\end{itemize}
\end{frame}

\begin{frame}
	\frametitle{Optimización Prematura}
	\begin{itemize}
		\item \textbf{Frase:} "La optimización prematura es la raíz de todo mal."
		\item \textbf{Explicación:} Aunque esta frase es comúnmente atribuida a Donald Knuth, la idea es que centrarse en optimizar el código desde el principio puede llevar a soluciones sobrecomplicadas y difíciles de mantener. La optimización debería ser una consideración posterior, una vez que el diseño básico esté sólidamente en su lugar.
	\end{itemize}
\end{frame}

\end{document}
