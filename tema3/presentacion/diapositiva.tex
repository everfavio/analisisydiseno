\documentclass[aspectratio=169]{beamer}
\usetheme{Warsaw}

\title{Ciclo de Vida del Desarrollo de Sistemas}
\author{Analisis y Diseño de Sistemas}
\date{\today}

\begin{document}

\begin{frame}
    \titlepage
\end{frame}

\begin{frame}{Índice}
    \tableofcontents
\end{frame}

\section{Etapas del análisis y desarrollo}
\begin{frame}{Estudio de Factibilidad}
    \begin{itemize}
        \item \textbf{Técnico}: Evaluar tecnologías, marcos de trabajo y recursos necesarios. Revisar si las infraestructuras existentes son compatibles.
        \item \textbf{Económico}: ROI, costos ocultos, evaluación del presupuesto y proyecciones financieras a corto y largo plazo.
        \item \textbf{Operativo}: Cómo se adaptará el sistema a las operaciones diarias. Evaluación de cambios en los procedimientos operativos y requisitos de formación del personal.
    \end{itemize}
\end{frame}

\section{Gestión de Requerimientos}
\begin{frame}{Gestión de Requerimientos}
    \begin{itemize}
        \item \textbf{Recopilación}: Entrevistas, cuestionarios, observación directa y retroalimentación.
        \item \textbf{Análisis}: Priorización de requerimientos, identificación de riesgos y restricciones.
        \item \textbf{Documentación}: Especificación de Requisitos del Sistema (SRS), casos de uso y diagramas.
        \item \textbf{Validación}: Aprobación por todas las partes interesadas, revisión por pares.
    \end{itemize}
\end{frame}

\section{Diseño}
\begin{frame}{Diseño}
    \begin{itemize}
        \item \textbf{Arquitectura}: Uso de patrones de diseño, descomposición modular y diagramas de arquitectura.
        \item \textbf{Componentes}: Definición de APIs, creación de microservicios o bibliotecas.
        \item \textbf{Datos}: Diseño de esquemas de base de datos, políticas de acceso y almacenamiento.
    \end{itemize}
\end{frame}

\section{Implementación y Pruebas}
\begin{frame}{Implementación y Pruebas}
    \begin{itemize}
        \item \textbf{Implementación}: Proceso de codificación, revisiones de código, control de versiones y documentación técnica.
        \item \textbf{Pruebas}: Pruebas unitarias para componentes individuales, pruebas de integración para módulos y pruebas de aceptación para el sistema en general.
        \item \textbf{Automatización}: Uso de CI/CD, herramientas como Jenkins o GitHub Actions para automatizar pruebas y despliegues.
    \end{itemize}
\end{frame}

\section{Bibliografía}
\begin{frame}{Bibliografía}
    \begin{itemize}
        \item "Feasibility Analysis of Projects" por Nasar, S.A.
        \item "Managing Software Requirements: A Use Case Approach" por Dean Leffingwell y Don Widrig.
        \item "Design Patterns: Elements of Reusable Object-Oriented Software" por Erich Gamma et al.
        \item "Clean Code: A Handbook of Agile Software Craftsmanship" por Robert C. Martin.
    \end{itemize}
\end{frame}

\end{document}
