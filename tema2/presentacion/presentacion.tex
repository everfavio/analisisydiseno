\documentclass[aspectratio=169]{beamer}
\usetheme{Warsaw}

\begin{document}

\title{Tema 2: Metodologías de Desarrollo de Software}
\author{Análisis y Diseño de Sistemas}
\date{\today}

\begin{frame}
	\titlepage
\end{frame}

% Introducción
\begin{frame}
	\frametitle{Introducción}
	\begin{itemize}
		\item Definición de Metodologías de Desarrollo
		\item Importancia de Elegir una Metodología Adecuada
		\item Conceptos Clave: Ciclo de Vida del Desarrollo de Software (SDLC), Stakeholders, ROI (Retorno de la Inversión)
	\end{itemize}
\end{frame}

% Enfoque de Desarrollo en Cascada
\begin{frame}
	\frametitle{Enfoque de Desarrollo en Cascada: Introducción}
	\begin{itemize}
		\item Modelo lineal y secuencial.
		\item Cada fase debe completarse antes de pasar a la siguiente.
		\item Conceptos Clave: Fases Secuenciales, Documentación Rigurosa, Modelo Predictivo
	\end{itemize}
\end{frame}

\begin{frame}
	\frametitle{Enfoque de Desarrollo en Cascada: Fases}
	\begin{itemize}
		\item Requisitos: Definir qué hará el sistema.
		\item Diseño: Especificar cómo se implementará.
		\item Implementación: Codificación del diseño.
		\item Verificación: Validación y pruebas.
		\item Mantenimiento: Soporte post-lanzamiento.
	\end{itemize}
\end{frame}

% Desarrollo Iterativo e Incremental
\begin{frame}
	\frametitle{Desarrollo Iterativo e Incremental: Introducción}
	\begin{itemize}
		\item Combina elementos de linealidad y ciclicidad.
		\item Iteraciones sucesivas a lo largo del ciclo de vida.
		\item Conceptos Clave: Iteraciones, Incrementos, Feedback Continuo
	\end{itemize}
\end{frame}

\begin{frame}
	\frametitle{Desarrollo Iterativo e Incremental: Fases Iterativas}
	\begin{itemize}
		\item Inicio: Definir los requisitos básicos y el diseño inicial.
		\item Iteración: Implementar, probar, mejorar.
		\item Finalización: Completar la implementación y verificar el sistema.
	\end{itemize}
\end{frame}

% Desarrollo Orientado a Prototipos
\begin{frame}
	\frametitle{Desarrollo Orientado a Prototipos: Introducción}
	\begin{itemize}
		\item Desarrollo de prototipos para probar conceptos.
		\item Ideal para requisitos poco claros.
		\item Conceptos Clave: Prototipo, Feedback del Usuario, Iteración Rápida
	\end{itemize}
\end{frame}

\begin{frame}
	\frametitle{Desarrollo Orientado a Prototipos: Tipos de Prototipos}
	\begin{itemize}
		\item Prototipos de lanzamiento rápido: Modelos iniciales para probar ideas.
		\item Prototipos evolutivos: Desarrollados y mejorados iterativamente.
	\end{itemize}
\end{frame}

% Scrum
\begin{frame}
	\frametitle{Scrum: Introducción}
	\begin{itemize}
		\item Marco de trabajo ágil.
		\item Divide el proyecto en ciclos llamados Sprints.
		\item Conceptos Clave: Sprint, Scrum Master, Product Backlog, Empirismo
	\end{itemize}
\end{frame}

% Kanban
\begin{frame}
	\frametitle{Kanban: Introducción}
	\begin{itemize}
		\item Basado en el sistema de producción Toyota.
		\item Visualización del flujo de trabajo.
		\item Conceptos Clave: Tablero Kanban, Work In Progress (WIP), Flujo de Valor
	\end{itemize}
\end{frame}

% Extreme Programming (XP)
\begin{frame}
	\frametitle{Extreme Programming (XP): Introducción}
	\begin{itemize}
		\item Enfoque en la excelencia técnica.
		\item Promueve la programación en pareja y la revisión de código.
		\item Conceptos Clave: Programación en Pareja, Integración Continua, Desarrollo Orientado por Pruebas (TDD)
	\end{itemize}
\end{frame}

% Conclusión
\begin{frame}
	\frametitle{Conclusión}
	\begin{itemize}
		\item Elección de la Metodología Adecuada
		\item Combinación de Metodologías
		\item Tendencias Futuras en Metodologías de Desarrollo
		\item Conceptos Clave: Agilidad, Escalabilidad, DevOps
	\end{itemize}
\end{frame}

\begin{frame}
	\frametitle{Bibliografía y Recursos Adicionales}
	\begin{itemize}
		\item "Scrum: The Art of Doing Twice the Work in Half the Time" - Jeff Sutherland
		\item "User Stories Applied" - Mike Cohn
		\item "The Lean Startup" - Eric Ries
		\item "Extreme Programming Explained" - Kent Beck
	\end{itemize}
\end{frame}

\begin{frame}
	\frametitle{Preguntas y Respuestas}
	\centering
	\large
	¡Gracias por su atención! \\
	¿Preguntas?
\end{frame}

\end{document}

